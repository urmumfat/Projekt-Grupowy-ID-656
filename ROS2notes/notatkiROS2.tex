\documentclass[12pt, a4paper]{article}

% --- Pakiety podstawowe ---
\usepackage[utf8]{inputenc}
\usepackage[T1]{fontenc}
\usepackage[polish]{babel}
\usepackage{geometry}
\geometry{margin=2.5cm}
\usepackage{xcolor}
\usepackage{enumitem}

% --- Konfiguracja Hyperref (Czysty spis treści) ---
\usepackage{hyperref}
\hypersetup{
	colorlinks=false,       % Wyłącza kolory linków
	pdfborder={0 0 0},      % Usuwa ramki wokół linków
	hidelinks               % Gwarantuje, że linki są czarne i bez podkreśleń
}

% --- Konfiguracja Terminala (Styl z Twojego zdjęcia) ---
\usepackage[most]{tcolorbox}
\tcbuselibrary{listings}

\newtcblisting{terminal}[1][]{
	arc=1pt,                % Lekkie zaokrąglenie rogów
	outer arc=1pt,
	colback=gray!10,        % Jasnoszare tło
	colframe=gray!30,       % Delikatna szara ramka
	boxrule=0.5pt,          % Cienka linia ramki
	listing only,
	enhanced,
	listing options={
		basicstyle=\small\ttfamily\color{black!80},
		breaklines=true,
		columns=fullflexible,
		keepspaces=true,
	},
	left=8mm,               % Miejsce na znak zachęty
	top=2mm,
	bottom=2mm,
	overlay={
		\node[anchor=west] at ([xshift=2.5mm]frame.west) 
		{\small\ttfamily\textcolor{orange!80!black}{\textbf{\$}}}; % Znak $ w kolorze pomarańczowo-brązowym
	},
	#1                      % Pozwala na dodatkowe opcje
}

% --- Styl sekcji (prosty i czarny) ---
\usepackage{titlesec}
%\titleformat{\section}{\Large\bfseries}{}{0em}{}[\titlerule]

\begin{document}
	
	\title{Notatki z Systemów Operacyjnych / ROS2}
	\author{Matrix}
	\date{\today}
	\maketitle
	
	% Spis treści będzie teraz całkowicie czarny i czysty
	\tableofcontents
	\newpage
	
	\section{Informacje wstępne}
	Dystrybucja ROS2: Kilted Kaiju
	\\
	Link do strony dystrybucji z poradnikami: https://docs.ros.org/en/kilted/Tutorials.html
	\\
	Komendy tu przedstawione są wykonywane na Linuxie w dystrybucji Ubuntu
	
	\section{Konfiguracja Środowiska}
	Po instalacji ROS2 przed rozpoczęcią pracy musimy ustawić najpier źródł skąd bierzemy instrukcjie tzn. 
	miejsce instalcji naszego oprogramowania.
	\\
	Żeby ustwaić źródł musimy użyć następującej komendy:
	\begin{terminal}
		source twoja_sciezka/ros/kilted/setup.bash
	\end{terminal}
	Żeby dodać to źródło na stałe korzystamy z tej komendy:
	\begin{terminal}
		echo "source twoja_sciezka/ros/kilted/setup.bash" >> ~/.bashrc
	\end{terminal}
	Dzięki następującej komendzie możemy sprawdzić czy nasze środowisko jest poprawnie skonfigurowane:
	\begin{terminal}
		printenv | grep -i ROS
	\end{terminal}

	\section{Węzły (Nodes)}
	Wykres ROS to sieć elementów ROS 2 przetwarzających dane jednocześnie. Obejmuje on wszystkie pliki wykonywalne i połączenia między nimi, gdyby je wszystkie zmapować i zwizualizować.
	\\
	\indent
	Każdy węzeł odpowiada za jeden moduł np. sterowanie silnikiem lub publikowania (ang. publishing) dane z LIDARA. Każdy node może wysyłać i odbierac dane z innych powiązanych z nim węzłów za pośrednictwem tzw. tematów (ang. topics), usług (ang.services), działań lub parametrów.
	
	
	
\end{document}